\documentclass{article}
\usepackage{graphicx}
\graphicspath{ {images/} }

%opening
\title{Mandatory assignment 01}
\author{Dennis Thinh Tan Nguyen}

\begin{document}

\maketitle
\newpage
\tableofcontents

\newpage
\section{Important Backend design choices}
During the design and development of Tingle V.4, many design choices have been made to improve the backend of application compared to Tingle V.3.
Some of the most important choices are described in the following subsections


\subsection{MVC architecture}
Tingle is utilising the MVC architectural design pattern to separate the system into three interconnected parts. This was previously introduced in Tingle V2 with a simple implementation of the MVC design pattern (see figure \ref{fig:TingleApp_ClassDiagramV02} for class diagram).\\\\As for Tingle V4 with the introduction of fragments and other implementation choices such as dependency injection, the MVC pattern enabled one to extend the model, controller or view in a clear way and keep complexity within each of their respected set of responsibility (see figure \ref{fig:TingleApp_ClassDiagramV08} for class diagram).




\subsection{Repository pattern with dependency injection}
To enable future extensions of the repository, a generic repository pattern has been adopted. The previous hardcoded thingsDB (TingleV2-3) in TingleActivity made it impracticable to extend or  even replace a given storage solution. Consequently, in Tingle V4, the following components have been introduced:
\begin{itemize}
	\item ThingsDatabase
	\item IRepository
	\item InMemoryRepository
	\item Entity
\end{itemize}
ThingsDatabase is a class that act as a repository in which activities  can perform CRUD operations on. It is provided as a singleton to TingleActivity. Other fragments can then request a reference to the instance. \\ThingsDatabase can be injected with any repository that implements the IRepository interface. The repository dependency is being handled by Dagger2 Dependency Framework\footnote{Dagger2 website: http://google.github.io/dagger/}, which is a fully static, compile-time third-party dependency injection framework.\\\\Currently, TingleApp V4 does not have a persistent way of storing database entities, but is, however, using the InMemoryRepository that stores items in the memory. This class is then injected to ThingsDatabase.
\subsubsection{IDs for Things/entities}
Another abstract class has also been introduced called Entity which serves as a superclass for the Thing Class.  The Entity class contains an ID which is used to uniquely identify a Thing, reduce future code duplication but also separate application logic from storage. A repository can take classes that are derived from Entity and store them.

\subsection{Fragment callbacks and eventhandlers}
Another important design choice is to use event handlers whenever a fragment is required to call back to the host activity. By way of example, when a user presses the add button, one may also want to update the view of a list containing items. One solution is to retrieve the host activity within a fragment and then call the appropriate method.\\\\This is not an effective solution as this will increase coupling but also expose unnecessary implementations. So, we use event handlers that a fragment can call. A host activity may then implement the event handlers and specify what will happen whenever an event is executed. In Tingle V4, this is used to change fragments within a shared container when in portrait mode and also update the things list when an item is added or deleted in landscape mode.

\section{User interface}
The current portrait UI of the main page can be seen on figure \ref{fig:mainpagePortrait} and the list page in portrait can be seen on figure  \ref{fig:listPortrait}. The landscape UI can be seen on figure \ref{fig:landscapeMode}.A complete manual is added to the appendix section \ref{manual}\\\\\textbf{Main page}\\The main page allows a user to add, look up or get an overview of most recent added item.\\\\\textbf{List Page}\\The list page gives the user a full overview of all of the added items as well as the option of deleting an item. 

\subsection{Usability design choices}
A couple of design choices has been made to improve the usability of Tingle. Every button, text boxes and interactive widgets have been, at best, placed at the bottom of the screen so the user can comfortably reach them.
 
\section{Functionality extensions to Tingle V3}
The following functionality has been added
\begin{itemize}
	\item Landscape split view
	\item Delete function

\end{itemize}
The landscape split view allows a user to see the list page as well as the main page on the same screen whenever the device is set to landscape mode. The delete function allows a user to delete a given stored item

\section{Test}
\subsection{Unit testing of model}
Automated unit tests have been performed on the model of Tingle. That is, testing the repository and database functionalities. Mockit framework has been used to create mocks when testing ThingsDatabse (See figure \ref{fig:testResultsV4} in appendix for test results of Tingle V4).

\subsection{Manual testing of View and application}
Taken the current state and complexity of Tingle V4, system tests and integration tests have been performed manually. However, one may want to use Android instrumentation tests to perform automated UI tests when the complexity of Tingle increases.  


\section{Bugs}
So far most bugs have been fixed in Tingle V4. 

\newpage
\section{Appendix}

\subsection{Manual of Tingle}\label{manual}
\textbf{Adding items}\\To add an item one can enter the name and location of the item in the two edit boxes and press the "Add new thing" button. The recently added header will then be updated with the newly added item.\\\\\textbf{Search for item}\\To search for a stored item, one can write the name of that given item in the "What" edit box (not case-sensitive) and press the "look up" button. A toast will then appear denoting its location.\\\\\textbf{Accessing list page}\\To get to the list page in portrait mode, one can press the "show all" button or flip the screen to go to landscape mode.\\\\\textbf{Delete item}\\To delete an item, one can press on an entry on the list to select the desired item. A toast will appear denoting what item is currently selected. By pressing the "Delete" button, the selected item will then be removed from storage.

\pagebreak

\subsection{Class Diagrams}
\begin{figure}[h!]
	\centering
	\includegraphics[scale=0.4]{"TingleApp_ClassDiagram V05"}
	\caption{Class Diagram of Tingle V2}
	\label{fig:TingleApp_ClassDiagramV02}
\end{figure}

\begin{figure}[h!]
	\centering
	\includegraphics[scale=0.4]{"TingleApp_ClassDiagram V08"}
	\caption{Class Diagram of Tingle V4}
	\label{fig:TingleApp_ClassDiagramV08}
\end{figure}

\pagebreak
\subsection{Test Results}
\begin{figure}[h!]
	\centering
	\includegraphics[scale=0.8]{"testResultsV4"}
	\caption{Unit test results of Tingle V4}
	\label{fig:testResultsV4}
\end{figure}


\pagebreak
\subsection{UI pictures}
\begin{figure}[h!]
	\centering
	\includegraphics[scale=0.7]{"portraitMode"}
	\caption{Main page in portrait mode}
	\label{fig:mainpagePortrait}
\end{figure}

\begin{figure}[h]
	\centering
	\includegraphics[scale=0.7]{"listPortrait"}
	\caption{List page in portrait mode}
	\label{fig:listPortrait}
\end{figure}

\begin{figure}[h!]
	\centering
	\includegraphics[scale=0.7]{"landscapeMode"}
	\caption{Landscape mode}
	\label{fig:landscapeMode}
\end{figure}



\end{document}
